\subsection{Magnetfelder von stromdurchgeflossenen Leitern}

Oersted (1820): Magnetische Wirkung eines geschlossenen Stromkreises
\break
$ \Rightarrow $ wenn $ I \neq 0 $: Drehmoment auf Kompassnadel, $ \vec{M}=\vec{\mu}\times \vec{B} $ \\
$ \Rightarrow $ $ \vec{B} $ in Ebene $ \perp $ Leiter \\
$ \Rightarrow $ Richtung (Vorzeichen, VZ) ist abhängig von Stromrichtung (Rechtsschraubregel)
BILD
\subsubsection{Bestimmung der Abstandsabhängigkeit}
$$ |\vec{B}| \sim \frac{I}{r}$$
$ \Rightarrow $ Beachte: "Quelle" des Magnetfeldes ist 1D $ \Rightarrow $ Abstandsabhängigkeit \\
quantitativ:
	\begin{align*}
		\boxed{|\vec{B}|=\frac{\mu_0}{2\pi} \cdot \frac{I}{r} }\\
		\mu_0 = \SI{4\pi e-7}{\newton\per\ampere^2} \text{magnetische Konstante}
	\end{align*}
	
BILD

$ \Rightarrow $ Bewegte Ladungen erzeugen ein Magnetfeld in Richtung: $ \frac{\vec{v}}{|\vec{v}|} \times \frac{\vec{r}}{|\vec{r}|}  $ \\
Bisher:
\begin{itemize}
	\item $ \vec{B} $-Feld übt Kraft auf bewegte Ladung aus
	\item Bewegte Ladung (Strom) erzeugt Magnetfeld
\end{itemize}
$ \Rightarrow $ Wechselwirkung zwischen Strömen muss existieren!

BILD

$ I_1 $ erzeugt $ \vec{B} $-Feld am Ort des Leiters $ L_2 $
$$ |\vec{B}_1| = \frac{\mu_0}{2\pi} \cdot \frac{I_1}{r} $$
$ \Rightarrow $ Kraft auf $ L_2 $

\begin{align*}
\vec{F}_{21} = L \cdot \vec{I}_2 \times \vec{B}_1\\
|\vec{F}_{21}| &= \frac{\mu_0}{2\pi} \cdot \frac{I_1 \cdot I_2}{r} \cdot L\\
&= \underline{\underline{\num{2e-7} \cdot \frac{I_1 \cdot I_2}{r} \cdot L}}
\end{align*}
$ \vec{F}_{21} $: anziehend bei parallelen Strömen! 
\subsubsection{Definition Ampere}
\paragraph{Elektrische Stromstärke: Ampere [A]}
Das Ampere ist die Stärke eines konstanten elektrischen Stromes, der, durch zwei parallele, gradlinige, unendlich lange und im Vakuum im Abstand von einem Meter voneinander angeordnete Leiter von vernachlässigbar kleinem, kreisförmigem Querschnitt fließend, zwischen diesen Leitern je einem Meter Leiterlänge die Kraft $ \SI{2e-7}{\newton} $ hervorrufen würde.\hfill \break

Ströme parallel:\\
BILD
$ I_1 \uparrow\downarrow I_2 $ \\
BILD

Elektrostatik: Gaußscher Satz: Zusammenhang zwischen Quellen-Verteilung und Feldstärke! \\
\indent $ \Rightarrow $ Normalkomponente des $ \vec{E} $-Feldes wichtig\\
Hier: Ströme als "Quelle" des magnetischen Feldes \\
"Umhüllung" der Ströme \underline{ohne} die Quelle zu "schneiden" nur mit geschlossenem Umlauf möglich (Quellen \underline{nicht} punktförmig!)\\
Beachte also: $ \displaystyle \oint\vec{B}d\vec{s} $
BILD
