\subsection{Überlagerung von Wellen} (in einer Dimension)\\
Superpositionsprinzip: Ungestörte Überlagerung mehrerer Wellen
$$ \Psi(\vec{r},t) = \sum_{i=1}^{N} \Psi_1(\vec{r},t)$$
\underline{Reflexion von Wellen $ \rightarrow $ stehende Wellen}\\
Experiment: Reflexion der Welle einer Pendelkette
\begin{itemize}
	\item Festes Ende: Phasensprung von $ \pi $ bei Reflexion
	\bild
	\item Freies Ende: Kein Phasensprung
	\bild
\end{itemize}
Überlegung von links- und rechts-laufender Welle\\
Gleiche Amplitude, gleiche Frequenz, Phasenverschiebung $ \varphi $
\begin{align*}
\Psi(x,t) &=\Psi_0 \cdot \cos(Kx-\omega t) \pm \Psi \cdot \cos(Kx+\omega t + \varphi)\\
\text{Additionstheroeme} \rightarrow &= \underline{\underline{2\cdot\Psi_0\cdot\cos(\omega t + \frac{\varphi}{2}) \cdot \cos(Kx+\frac{\varphi}{2})  } }
\end{align*}
Faktorisieren der Orts- und Zeitabhängigkeit:\\
Für alle $ t $ gibt es Punkte x; bei denen  $ \Psi(x,t) = 0 $: "Knoten"\\
Betrachte Orts-Teil: Festes Ende: $ \varphi = \pi $\\
Knoten: $ \underbrace{\cos(Kx+\frac{\varphi}{z})}_{\Rightarrow \text{Knoten bei } x_0, x_0+\frac{\pi}{2}, x_0+\frac{3\pi}{2}...} = 0 \Rightarrow Kx+\overbrace{\frac{\varphi}{2}}^{\pi/2}= \frac{(2\mathbb{N}+1)}{2} \cdot \pi $\\ \break
Resonanz-Situation: $  \underset{K=\frac{2\pi}{2}}{K} \cdot x = \frac{(2\mathbb{N}+1)}{2} \cdot \pi $\\
\begin{center}
	Resonanz: $ \underset{\text{Länge}}{L} = \frac{\mathbb{N}}{2} \cdot \lambda $\\
\underline{$ \Rightarrow \lambda = \frac{2}{\mathbb{N}}\cdot L $}\\
\underline{$ \mathbb{N}\cdot\frac{\lambda}{2} = L $}\\
\end{center}
 \bild
 Loses Ende: $ \varphi = 0 $!
 \begin{align*}
 \Rightarrow &\text{ Resonanz: } \frac{2\mathbb{N}+1}{2} \cdot \lambda = L\\
 &\lambda = \frac{4 \cdot L}{(2\mathbb{N}-1)}
 \end{align*}
 \bild
 Mikrowelle:
 \begin{align*}
 d_{Knoten} &= \frac{\lambda}{2} \approx \SI{10}{\centi\meter}\\
 \nu & = \frac{c}{\lambda} = \frac{\SI{3e8}{\meter\per\second}}{\SI{0,2}{\meter}} = \SI{1,5e9}{\per\second}\\
 &=\underline{\SI{1,5}{\giga\hertz}}
 \end{align*}
 \bild
 \bild