\documentclass[11pt]{article}

\usepackage{amssymb}
\usepackage{amsthm}
\usepackage{amsmath}
\usepackage{mathtools}
\usepackage[utf8]{inputenc}
\usepackage{mathabx}
%\usepackage{framed}
\usepackage{booktabs}
\usepackage{hyperref}
\usepackage{txfonts}

\hypersetup
	{ 
		colorlinks=true,       % false: boxed links; true: colored links
%		hidelinks,
		linkcolor=blue,          % color of internal links (change box color with linkbordercolor)
		citecolor=green,        % color of links to bibliography
		filecolor=magenta,      % color of file links
		urlcolor=cyan,           % color of external links
		linkbordercolor	= {1 0 0},
		citebordercolor	= {0 1 0},	
		urlbordercolor	= {0 1 1}
	}


%\usepackage{fontspec}
%\setmainfont{Clear Sans}
%\newfontfamily{\clearsans}{Clear Sans}

\newcommand{\definition}{\\ \textbf{Definition:} \hspace{1cm} }

\newcommand*{\QEDA}{\hfill\ensuremath{\blacksquare}}%
\newcommand*{\QEDB}{\hfill\ensuremath{\square}}%


\begin{document}
	\title{Kapitel 14 - Statische elektrische Felder}
		\author
		{Johannes Bilk
			\\
			{\small 	\texttt{me@talachem.de}}
		}
		\date{\today}
	\maketitle
	\tableofcontents
	\setcounter{section}{13} %Hier f\"{a}ngt die Nummerierung an.
	
	\newpage
	
\section{Statische Elektrische Felder}	
	\subsection{ Elektrische Ladungen }	
	
	$\rightarrow$ Ab dem 17. Jahrhundert: Ursache für "elektrische Ph\"{a}nomene"; "neuartiger Stoff", elektrische Ladung
	
		\subsubsection{ Reibungselektriziz\"{a}t }
		
			\begin{itemize}
			\item Zwei Arten von "elektrischen Zust\"{a}nden" sind erzeugbar:
				\begin{itemize}
				\item Gleichartige Zust\"{a}nde $\implies$ Abstoßung
				\item Ungleichartige Zust\"{a}nde $\implies$ Anziehung
			\end{itemize}
			\item Carles Du Fay (1730): positiv/negativ elektrische Ladung
			\item Benjamin Franklin (1750): Über-/Unterschuss an "elektrischen Fluiden"
			\item Lichtenberg (1778): Zuordnung der Polari\"{a}t
			\end{itemize}
			
			\fbox{\begin{minipage}{19em}
			Hargummi stab: reiben mit Pelz, Wolle: -\\
			Glas, Plexiglas: reiben mit Seide: +
			\end{minipage}}
			\linebreak\\
			Reibezeug: entgegengesetzte Polarit\"{a}t
			$\implies$ Ladungstrennung, nicht etwa Ladungserzeugung.
			\linebreak\\
			Grunds\"{a}tzliches Messprinzip: Elektroskop: \hfill \\
			
			$\rightarrow$ Elektrometer $\rightarrow$ quantitative Messung
			\begin{itemize}
				\item "L\"{o}ffeln"; d.h. portionsweise Übertragung von Ladungen ist mglich
				\item Elektropendel: $\implies$ periodisches Umladen eines "Kugelpendel"
			\end{itemize}
			
			\subsubsection{Ladung ist eine skalare Grße } Einheit 1C = 1 Coulomb, SI
				\begin{itemize}
					\item Zu jedem geladenen Elementarteilchen gibt es ein Elementarteilchen mit entgegengesetzter Ladung ($\rightarrow$ Ladungssymmetrie)
					\item Die Gesamtladung eines abgeschlossenen Systems bleibt erhalten ($\rightarrow$ Ladungserhaltung)
					\item Beispiel: Produktion eines $ e^+e^- $-Paares; $ E_\gamma \geq $ 1,02 MeV
				\end{itemize}
				
				\newpage
				
				\noindent Nachweis: Blasenkammer im Magnetfeld:  \hfill \\
				Umkehrung: "Zerstrahlung" von Positronen; $E=m\cdot c^2$
				\begin{itemize}
					\item Ladungtr\"{a}ger haben stets eine Masse
					\item Ladung kann nicht (im Gegensatz zur Masse) in Energie umgewandelt werden, bleibt auch bei Zerfallsprozessen erhalten.
					\item Quantisierung der Ladung: Alle in der Natur vorkommenden Ladungen sind ganzzahlige Vielfache der Elementarladung: $e_0:=1,602\cdot10^{-19}C; 1C=1AS$
				\end{itemize}
				\subparagraph{Beispiele von Ladungen}
				\begin{itemize}
					\item Neutral: $\gamma, \nu, n$
					\item einfach geladen: $e^-,e^+,p, \bar{p}$
					\item zweifach geladen:: $He_2(2^+,Z:2)$
				\end{itemize}	
				
\newpage

\subsubsection{ Quarks }	
\paragraph{Seit 60er Jahre}
Nukleonen bestehen aus Quarks, diese haben "drittelzahlige Ladungen"
\\
Up-Quarks:$u:+\frac{2}{3	}e_0$
\\
Down-Quarks:$d:-\frac{1}{3}e_0$
\\
Proton:$2u+d: 1\cdot e_0$
\\
Neutron:$u+2d: 0\cdot e_0$
\\

Quarks treten immer in 2er- oder 3er- Kombinationen auf.
\\

\subsubsection{Entdeckung und Bestimmung der Elementarladung}

Robert Andrews Millikan(1868-1953): Öltrpfchenversuch ($\rightarrow$ Anf\"{a}ngerpraktikum)


\subsection{Kr\"{a}fte zwischen Ladungen und das Coulomb-Gesetz} 

Charles-Augustin de Coulomb (1736-1806)

1785: Messung der Kraft zwischen zwei Ladungen als Funktion des Abstands mit Hilfe einer Torsionswaage \hfill \\

$$ \boxed{\vec{F_{12}} = f\cdot\frac{Q_1\cdot Q_2}{r^2_{12}}\cdot\frac{\vec{r_{12}}}{|\vec{r_{12}}|} = f\cdot\frac{Q_1\cdot Q_2}{r^2_{12}}\cdot \hat{r}_{12}} $$

\noindent F ist definiert durch die Definition der Ladungseinheit:
\\
Internationales Messsystem (SI): $f=\frac{1}{4\pi\epsilon_0}$
\\
$\epsilon_0=8,854\cdot10^{-12}\frac{(As)^2}{Nm^2}$
\\
ist Dielektrizit\"{a}tskonste des Vakuums oder elektrische Feldkonstante
\\
$Q_1\cdot Q_2 > 0:$ Abstoßung
\\
$Q_1\cdot Q_2 < 0:$ Anziehung

\subsection{Potenzielle Energie einer Ladungsverteilung}

...

\subsection{Erzeugung el. Felder durch Ladungen}

	\subsubsection{Feld einer Punktladung:}
	
	\begin{align*}
		\vec{F} &= \frac{1}{4\pi\epsilon_0} \cdot \frac{q_1 q_2}{ |\vec{r}_{12}|^2 } \cdot \hat{r_{12}} \\
					&=q_1 \underbrace{ \cdot \frac{1}{4\pi\epsilon_0} \cdot \frac{q_2}{ |\vec{r}_{12}|^2 } \cdot \hat{r_{12}}  }_{\text{Feld von }q_2} \\
					&=q_1 \vec{E}(\vec{r})
	\end{align*}

\end{document}
