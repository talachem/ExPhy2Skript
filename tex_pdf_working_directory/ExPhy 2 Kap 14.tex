\documentclass[11pt]{article}

\usepackage[utf8]{inputenc}
\usepackage[T1]{fontenc}

\usepackage{amssymb}
\usepackage{amsthm}
\usepackage{amsmath}
\usepackage{mathtools}
\usepackage{mathabx}
%\usepackage{framed}
\usepackage{booktabs}
\usepackage{hyperref}
\usepackage{txfonts}
\usepackage{siunitx}



\hypersetup
	{ 
		colorlinks=true,       % false: boxed links; true: colored links
%		hidelinks,
		linkcolor=blue,          % color of internal links (change box color with linkbordercolor)
		citecolor=green,        % color of links to bibliography
		filecolor=magenta,      % color of file links
		urlcolor=cyan,           % color of external links
		linkbordercolor	= {1 0 0},
		citebordercolor	= {0 1 0},	
		urlbordercolor	= {0 1 1}
	}


%\usepackage{fontspec}
%\setmainfont{Clear Sans}
%\newfontfamily{\clearsans}{Clear Sans}

\newcommand{\definition}{\\ \textbf{Definition:} \hspace{1cm} }

\newcommand*{\QEDA}{\hfill\ensuremath{\blacksquare}}%
\newcommand*{\QEDB}{\hfill\ensuremath{\square}}%


\begin{document}
	\title{Kapitel 14 - Statische elektrische Felder}
		\author
		{Johannes Bilk
			\\
			{\small 	\texttt{me@talachem.de}}
		}
		\date{\today}
	\maketitle
	\tableofcontents
	\setcounter{section}{13} %Hier f\"{a}ngt die Nummerierung an.
	
	\newpage
	
\section{Statische Elektrische Felder}	
	\subsection{ Elektrische Ladungen }	
	
	$\rightarrow$ Ab dem 17. Jahrhundert: Ursache für "elektrische Ph\"{a}nomene"; "neuartiger Stoff", elektrische Ladung\\
\includegraphics{skizzen/14/14_1B0}
		\subsubsection{ Reibungselektriziz\"{a}t }
		
			\begin{itemize}
			\item Zwei Arten von "elektrischen Zust\"{a}nden" sind erzeugbar:
				\begin{itemize}
				\item Gleichartige Zust\"{a}nde $\implies$ Abstoßung
				\item Ungleichartige Zust\"{a}nde $\implies$ Anziehung
			\end{itemize}
			\item Carles Du Fay (1730): positiv/negativ elektrische Ladung
			\item Benjamin Franklin (1750): Über-/Unterschuss an "elektrischen Fluiden"
			\item Lichtenberg (1778): Zuordnung der Polari\"{a}t
			\end{itemize}
			
			\fbox{\begin{minipage}{19em}
			Hargummi stab: reiben mit Pelz, Wolle: -\\
			Glas, Plexiglas: reiben mit Seide: +
			\end{minipage}}
			\linebreak\\
			Reibezeug: entgegengesetzte Polarit\"{a}t
			$\implies$ Ladungstrennung, nicht etwa Ladungserzeugung.
			\linebreak\\
			\subparagraph {Grunds\"{a}tzliches Messprinzip: Elektroskop:} \hfill \\
			
			\includegraphics{skizzen/14/14_1B1}
			
			$\rightarrow$ Elektrometer $\rightarrow$ quantitative Messung
			\begin{itemize}
				\item "L\"{o}ffeln"; d.h. portionsweise Übertragung von Ladungen ist mglich
				\item Elektropendel: $\implies$ periodisches Umladen eines "Kugelpendel"
			\end{itemize}
			
			\subsubsection{Ladung ist eine skalare Gr\"{o}\ss{}e } Einheit 1C = 1 Coulomb, SI
				\begin{itemize}
					\item Zu jedem geladenen Elementarteilchen gibt es ein Elementarteilchen mit entgegengesetzter Ladung ($\rightarrow$ Ladungssymmetrie)
					\item Die Gesamtladung eines abgeschlossenen Systems bleibt erhalten ($\rightarrow$ Ladungserhaltung)
					\item Beispiel: Produktion eines $ e^+e^- $-Paares; $ E_\gamma \geq $ 1,02 MeV
				\end{itemize}
				
				\includegraphics{skizzen/14/14_1B2}
				
				\newpage
				
				\noindent Nachweis: Blasenkammer im Magnetfeld:  \hfill \\
				Umkehrung: "Zerstrahlung" von Positronen; $E=m\cdot c^2$
				\begin{itemize}
					\item Ladungtr\"{a}ger haben stets eine Masse
					\item Ladung kann nicht (im Gegensatz zur Masse) in Energie umgewandelt werden, bleibt auch bei Zerfallsprozessen erhalten.
					\item Quantisierung der Ladung: Alle in der Natur vorkommenden Ladungen sind ganzzahlige Vielfache der Elementarladung: $e_0:=1,602\cdot10^{-19}C; 1C=1AS$
				\end{itemize}
				\subparagraph{Beispiele von Ladungen}
				\begin{itemize}
					\item Neutral: $\gamma, \nu, n$
					\item einfach geladen: $e^-,e^+,p, \bar{p}$
					\item zweifach geladen:: $He_2(2^+,Z:2)$
				\end{itemize}	
				

\subsubsection{ Quarks }	
\paragraph{Seit 60er Jahre}
Nukleonen bestehen aus Quarks, diese haben "drittelzahlige Ladungen"

\includegraphics{skizzen/14/14_1B5}
\\
Up-Quarks:$u:+\frac{2}{3	}e_0$
\\
Down-Quarks:$d:-\frac{1}{3}e_0$
\\
Proton:$2u+d: 1\cdot e_0$
\\
Neutron:$u+2d: 0\cdot e_0$
\\

Quarks treten immer in 2er- oder 3er- Kombinationen auf.
\\

\subsubsection{Entdeckung und Bestimmung der Elementarladung}

Robert Andrews Millikan(1868-1953): Öltrpfchenversuch ($\rightarrow$ Anf\"{a}ngerpraktikum)


\subsection{Kr\"{a}fte zwischen Ladungen und das Coulomb-Gesetz} 

Charles-Augustin de Coulomb (1736-1806)

1785: Messung der Kraft zwischen zwei Ladungen als Funktion des Abstands mit Hilfe einer Torsionswaage \hfill \\

\includegraphics{skizzen/14/14_2B0}

$$ \boxed{\vec{F_{12}} = f\cdot\frac{Q_1\cdot Q_2}{r^2_{12}}\cdot\frac{\vec{r_{12}}}{|\vec{r_{12}}|} = f\cdot\frac{Q_1\cdot Q_2}{r^2_{12}}\cdot \hat{r}_{12}} $$

\noindent F ist definiert durch die Definition der Ladungseinheit:
\\
Internationales Messsystem (SI): $f=\frac{1}{4\pi\epsilon_0}$
\\
$\epsilon_0=8,854\cdot10^{-12}\frac{(As)^2}{Nm^2}$
\\
ist Dielektrizit\"{a}tskonste des Vakuums oder elektrische Feldkonstante
\\
$Q_1\cdot Q_2 > 0:$ Abstoßung
\\
$Q_1\cdot Q_2 < 0:$ Anziehung
\\
\subparagraph{Coulomb-Kraft}

\hfill\\

\boxed{\vec{F}_{12} = \frac{1}{4\pi\epsilon_0}\cdot\frac{Q_1\cdot Q_2}{r^2_{12}}\cdot\hat{r}_{12}}

\hfill\\

\noindent\fbox{%
\parbox{\textwidth}{%
	1 Coulomb ist diejenige elektrische Ladung, die eine gleich große Ladung im Abstand von 1m mit der Kraft von $8,9874\cdot10^9N$ abstößt	
}%
}

\subparagraph{Analogie Gravitation}: $\vec{F} = -\gamma\frac{m_1\cdot m_2}{r^2}\cdot\hat r$\\

Vergleiche Gravitation und Coulombkraft zwischen Elektron und Proton:

$$ |\vec{F}_C| = \frac{1}{4\pi\epsilon_0}\cdot\frac{|q_p|\cdot|q_e|}{r^2}= 2,3\cdot10^{ -28 } \frac{N}{r[m]^2} $$
$$ |\vec{F}_G|= \gamma\cdot m_e\cdot m_p  = 9,71\cdot10^{-68} \frac{N}{r[m]^2}$$

$$ \implies \frac{|F_G|}{|F_C|} = 4,2\cdot10^{-40} $$

\subparagraph{Wechselwirkung zwischen mehreren Ladungen}\\

\hfill\\

\includegraphics{skizzen/14/14_2B1}

\hfill\\

Die einzelnen Kräfte überlagern sich ungestört, (ungestörte Superposition)!\\

Kraft auf $$ Q_3: \vec{F}_3 = \bigg[\frac{Q_1\cdot Q_3}{r^2_{13}}\cdot\hat{r}_{13}+\frac{Q_2\cdot Q_3}{r^2_{23}}\cdot\hat{r}_{23}\bigg]$$ 


\subsection{Potenzielle Energie einer Ladungsverteilung}


\begin{align*}
\displaystyle W_{12} &= -\frac{1}{4\pi\epsilon_0}\cdot\int_{\infty}^{r} \frac{Q_1\cdot Q_2}{V^2}  \mathrm{d}V\\
&=\frac{1}{4\pi\epsilon_0} \big[\frac{Q_1\cdot Q_2}{V}\big]^{12}_\infty = \frac{1}{4\pi\epsilon_0}\cdot \frac{Q_1\cdot Q_2}{V_{12}}
\hfill\\
W_{1,2,3} &= \frac{1}{4\pi\epsilon_0}(\frac{Q_1\cdot Q_3}{r_{13}}+\frac{Q_1\cdot Q_2}{r_{12}}+\frac{Q_2\cdot Q_3}{r_{23}})
\end{align*}

\newline\\
\newline\\
\linebreak\\
\hfill\\
Anzahl an Summanden = Anzahl an Paaren
\newline\\

\boxed{W = \bigg[\frac{1}{2}\sum_{i=1}^{N} \sum_{j≠i=1}^{N} \frac{Q_i\cdot Q_j}{r_{ij}}\bigg]\cdot \frac{1}{4\pi\epsilon_0}}

\newline\\
\linebreak\\
\hfill\\

$\implies$ Aufsummieren auch unendlicher Ensembles möglich, wenn die Reihe konvergiert. 

\paragraph{Betrachte Kraft auf Probeladung in homogen geladener Kugel}

\hfill\\

\includegraphics{skizzen/14/14_3B0}

\hfill\\

Für beliebe Räumlichelemente (und damit auch Flächenelemente) gilt:

$q_1 \propto \mathrm{d}A_1 \propto r_1^2$

$q_1 \propto \mathrm{d}A_1 \propto r_1^2$

Geometrie $\implies \frac{q_1}{q_2} = \frac{r_1^2}{r_2^2}$

Annahme: Kraft \propto \frac{1}{r^n}

$\vert\vec{F}_1\vert = \frac{1}{4\pi\epsilon_0}\cdot\frac{q\cdot q_1}{r_1^n}\\
\vert\vec{F}_2\vert = \frac{1}{4\pi\epsilon_0}\cdot\frac{q\cdot q_2}{r_2^n}$\\

Geometrie einsetzen: 

$\frac{|\vec{F}_1|}{|\vec{F}_2|} = \frac{q_1}{r_1^n}\cdot \frac{r_2^n}{q_2} \overset{!}{=} 1 \implies n = 2$

Gesamtkraft verschwindet nur wenn |\vec{F}| \propto \frac{1}{r^2}

\subsection{Erzeugung el. Felder durch Ladungen}

	\subsubsection{Feld einer Punktladung:}
	
	\begin{align*}
		\vec{F} &= \frac{1}{4\pi\epsilon_0} \cdot \frac{q_1 q_2}{ |\vec{r}_{12}|^2 } \cdot \hat{r_{12}} \\
					&=q_1 \underbrace{ \cdot \frac{1}{4\pi\epsilon_0} \cdot \frac{q_2}{ |\vec{r}_{12}|^2 } \cdot \hat{r_{12}}  }_{\text{Feld von }q_2} \\
					&=q_1 \vec{E}(\vec{r})
	\end{align*}
	\begin{itemize}
		\item Felder einer Punktladung sind Zentralfelder mit Kugelsymmetrie
		\item Konvention: Feldlinien führen von positiver zu negativer Ladung
	\end{itemize}
	\begin{center}
		\includegraphics[width=0.7\linewidth]{skizzen/14/14_4B0}
	\end{center}

	$ \Rightarrow $ \underline{Punktladungsfelder sind inhomogen!}
	
	\subsubsection{Feld einer Verteilung von Punktladungen}
	N Ladungen bei $ \vec{r_i} $ 
	$$ \vec{E_i}(\vec{r}) = \frac{1}{4\pi\epsilon_0} \cdot \frac{q_i}{ | \vec{r} - \vec{r_i} |^2 } \cdot \frac{ \vec{r} - \vec{r_i}  }{ | \vec{r} - \vec{r_i}| }  $$
	Ungestörte Superposition:
	$$  \vec{E}(\vec{r}) = \frac{1}{4\pi\epsilon_0} \cdot {\displaystyle \sum_{i=1}^{N} \frac{q_i}{ | \vec{r} - \vec{r_i} |^2 } } \cdot \frac{ \vec{r} - \vec{r_i}  }{ | \vec{r} - \vec{r_i}| } $$
	\begin{minipage}{\textwidth}
		
		2 Ladungen, q ; -q : Feld eines Dipols 
		\begin{center}
			\includegraphics[width=0.4\linewidth]{skizzen/14/14_4B1}
		\end{center}
	
		2 Ladungen: q ; q 
		\begin{center}
			\includegraphics[width=0.4\linewidth]{skizzen/14/14_4B2}
		\end{center}
	
	\end{minipage}
	
	\begin{minipage}{\textwidth}
	
		Beispiele für "natürliche Dipole": \\
		\begin{enumerate}
			\item Neutrales Atom im homogenen $ \vec{E} $-Feld  
			\begin{center}
				\includegraphics[width=0.9\linewidth]{skizzen/14/14_4B3}
			\end{center}
	
			\item Polare Molekühle mit permanentem Dipolmoment
			\begin{center}
				\includegraphics[width=0.4\linewidth]{skizzen/14/14_4B4}
			\end{center}
		
		\end{enumerate}
	
	\end{minipage}
	
	\subsubsection{Leiter im el. Feld und Influenz}
	Leiter: Ladungen sind \underline{frei} beweglich  \\
	Isolator: Ladungen sind ortsfest
	\begin{enumerate}
		\item $ \vec{E} = 0 $ im Inneren des Leiters 
		\begin{center}
			\includegraphics[width=0.4\linewidth]{skizzen/14/14_4B6}
		\end{center} 
		falls $ \vec{E} \neq 0 $: $ \vec{F} = q\vec{E} $ verschiebt Ladung bis $ \vec{E} = 0 $ ! 
		\item Es folgt, sich bei einem Leiter die (Netto-)Ladungen \underline{immer} an der Oberfläche befinden $ \Rightarrow $ Flächenladungsdichte $ \boxed{\sigma = \frac{dQ}{dA}} $
		\item $ \vec{E} $ immer $ \perp $ auf Leiteroberfläche
		\begin{center}
			\includegraphics[width=0.4\linewidth]{skizzen/14/14_4B7}
		\end{center}
		(falls $ \vec{E}_{\shortparallel} \neq 0 $: Verschiebung der Ladung bis $ \vec{E}_{\shortparallel} = 0 $ !)

	\end{enumerate}
	\paragraph{\underline{Influenz}:} Räumliche Ladungstrennung in el. Leitern durch äu\ss{}eres $ \vec{E} $-Feld (Kontaktlos!), so dass das Innere des Leiters Feldfrei ist! 
	

		
	\subsection{Kontinuierliche Ladungsverteilung}
	
		Betrachte Ladungsverteilung über endliches Volumen $ V = {\displaystyle \int_V dV} $
		\begin{center}
			\hypertarget{ref1}{}
			\includegraphics[width=0.5\linewidth]{skizzen/14/14_5B0}
			(\textasteriskcentered)
		\end{center}
		Ladungsdichte: $ \rho(\vec{r}) = \frac{dq(\vec{r})}{dV} $ \\
		Gesamtladung: $ Q = {\displaystyle\int_V dq} = {\displaystyle\int_V} \rho(\vec{r}) dV  $ \\
		Flächenladungsdichte: $ \sigma = \frac{dq}{dA} $ \\
		
		Integral über geschlossene Oberfläche: $ A = {\displaystyle\oint_A} dA $ \\
		1-dim Ladungsdichte: $ \lambda = \frac{dq}{dl} $ \\
		Länge $ l = {\displaystyle\int_l dl'} $ \\
		für \hyperlink{ref1}{(\textasteriskcentered)} :
		\begin{align*}
		d\vec{E}(\vec{r}) &= \frac{dq}{| \vec{r}-\vec{r}' |^3} \cdot (\vec{r}-\vec{r}') \cdot \frac{1}{4\pi\epsilon_0} \\
		\vec{E}(\vec{r}) &= \frac{1}{4\pi\epsilon_0} {\displaystyle\int_V dV} \left\{ \frac{dq}{dV} \cdot \frac{(\vec{r}-\vec{r}')}{| \vec{r}-\vec{r}' |^3} \right\} \\
		&=\underline{\frac{1}{4\pi\epsilon_0} {\displaystyle\int_V dV} \left\{ \frac{ \rho(\vec{r}) }{| \vec{r}-\vec{r}' |^3} \cdot (\vec{r}-\vec{r}') \right\}}
		\end{align*}
		
		\paragraph{Beispiel:} unendlich langer geladener Draht \\
		\begin{center}
			\includegraphics[width=0.5\linewidth]{skizzen/14/14_5B1}
		\end{center}
		\begin{align*}
		\frac{dq}{dx} = \lambda \text{ (lin. Ladungsdichte) } \\
		\text{Symmetrie: } E_x = E_z = 0 \\
		E_y = E \cdot \cos(\theta) \\
		dE_y& = | d\vec{E} | \cdot \cos (\theta) \\
		dE_y &= \frac{1}{4\pi\epsilon_0} \cdot 2 \cdot \frac{\lambda \cdot dx}{r^2 (\theta)} \cdot \cos (\theta) \hspace{1cm} ; \cos (\theta) = \frac{a}{r}&& \\
		&= \frac{1}{2\pi\epsilon_0} \cdot \lambda \cdot dx  \cdot \underline{\cos^2\theta} \cdot \cos\theta \\
		\hfill \\
		\tan\theta = \frac{x}{a} \Rightarrow \frac{dx}{d\theta} = \frac{a}{\cos^2\theta} \hspace{1cm} \Rightarrow dx = \underline{\frac{a}{\cos^2\theta} d\theta}
		\end{align*}
		
		\begin{align*}
		\Rightarrow dE_y &= \frac{1}{2\pi\epsilon_0} \cdot \lambda \cdot \frac{\cos\theta}{a} \cdot d\theta \\
		E_y &= \frac{1}{2\pi\epsilon_0} \cdot \frac{\lambda}{a} \cdot { \displaystyle\int_{0}^{\pi/2} \cos\theta d\theta } = \underline{\underline{\frac{1}{2\pi\epsilon_0} \cdot  \frac{\lambda}{a} }}
		\end{align*}

		
	
\subsection{Elektrischer Fluss und Satz von Gauß}
	
\hfill\\

\begin{wrapfigure}\centering\includegraphics{skizzen/14/14_6B0}\end{wrapfigure}
		
Zusammenhang zwischen "Elektrischem Fluss" (Feldliniendurchsatz) durch eine Oberfläche und der eingeschlossenen Ladung.

$\implies$ Allgemeinere Formulierung des Coulomb-Gesetzes

\subsubsection{Definition:} Fluss $\phi$ eines Vektorfeldes $\vec{E}$ durch einen Fläche A:

\hfill\\

\includegraphics{skizzen/14/14_6B1}

\\

$\phi= \int_A \vec{E}\cdot d\vec{A}$

\hfill\\

\begin{align*}
	$d\vec{A}$:&Richtung \perp Fläche (nach Außen)\\
	&Richtung der Flächennormale
\end{align*}
Betrag dA: Größe der Fläche

\subparagraph{Spezialfälle}

\hfill\\

$\vec{E}-homogen$
$\vec{E}\cdot d\vec{A}= E\cdot dA\cdot \cos\alpha$

\begin{itemize}
	\item $\alpha=0°: \vec{E}\parallel d\vec{A}: \vec{E}\cdot d\vec{A}=E\cdot dA$\\
	\item $\alpha=90°: \vec{E}\perp d\vec{A}: \vec{E}\cdot d\vec{A}=0$
\end{itemize}

\subsubsection{Gauß'scher Satz}

\boxed{$\displaystyle\phi=\oint_{\underset{geschlossen}{A}}\vec{E}\cdot d\vec{A}=\frac{Q_{eingeschlossen}}{\epsilon_0}$}

\hfill\\

Der elektrische Fluss durch einer belieben geschlossenen Oberfläche hängt weder von der Form der Oberfläche, noch von der Ladungsverteilung $\varrho(\vec{r})$ ab, sondern nur von der eingeschlossenen Gesamtladung Q.

\hfill\\

\subparagraph{Mathematisch gilt:}

\hfill\\

\begin{align*}
	$\oint_{A}\vec{E}\cdot d\vec{A}$&$=\int_{V} div\cdot \vec{E}\cdot dV$
	&$=\int_{V}\vec{\nabla}\cdot\vec{E}\cdot dV$
\end{align*}

\hfill\\

\begin{align*}
	$\implies \int_{V}\vec{\nabla}\vec{E}dV$&$=\frac{1}{\epsilon_0}\int_{V}\varrho(\vec{r})dV$
	&$=\vec{\nabla}\vec{E}=\frac{\varrho(\vec{r})}{\epsilon_0}$
\end{align*}


\hfill\\

Die Ladungsverteilung im Raum ist die lokale Quelle ($\varrho(\vec{r})>0$) bzw. Senke ($\varrho(\vec{r})<0$) des elektrischen Feldes.

\subsubsection{Beispiele}\\

(i)Feld einer Punktladung\\
\includegraphics{skizzen/14/14_6B2}\\
\begin{itemize}
	\item Geeignete Wahl von A: Kugeloberfläche
	\item Symmetrie: $\vec{E}(\vec{r})=E(r)\cdot\hat e_r$
	\item $\implies \vec{E}\parallel d\vec{A}$
\end{itemize}

\hfill\\

\begin{align*}
	$\phi=\oint_{A}\vec{E}(\vec{r})\cdot\mathrm{d}\vec{A}$&$= \oint_{A}E(r)\cdot\hat{e}_r\cdot\mathrm{d}\vec{A}$
	&$=\oint_{A}E(r)\cdot\mathrm{dA}$
	&$= E(r)\cdot\oint_{A}\mathrm{dA}$
	&$= E(r)\cdot4\pi r^2$
\end{align*}

\hfill\\

Gauß: $\phi\overset{!}{=} \frac{Q}{\epsilon_0}=E(r)\cdot4\pi r^2 \implies \underline{E(r)=\frac{1}{4\pi\epsilon_0}\cdot\frac{Q}{r^2}}$

\hfill\\

(ii) Ladung auf beliebig geformten \underline{Leitern}

\hfill\\

\includegraphics{skizzen/14/14_6B3}

\hfill\\

$\phi=\oint_{A}\vec{E}d\vec{A}=\frac{Q}{\epsilon_0}=0$

\hfill\\

(iii) Feld einer \underline{leitenden} Kugel mit Ladung Q: (Ladung auf der Oberfläche)

\hfill\\

\includegraphics{skizzen/14/14_6B4}

\hfill\\

$ \vec{E}(\vec{r})= E(\vec{r})\cdot\hat{e}_r $

$ r<R: E=0 $

$ r>R: \phi=\oint_{A}E(r)\mathrm{dA}=E(r)\cdot4\pi r^2 $

$ \phi=\frac{Q}{\epsilon_0} $

$ \implies \underline{E(r)= \frac{1}{4\pi\epsilon_0}\cdot\frac{Q}{r^2}} $

\hfill\\

\includegraphics{skizzen/14/14_6B5}

\hfill\\

(iv) Feld einer homogen geladenen Kugel

\hfill\\

\includegraphics{skizzen/14/14_6B6}

\hfill\\

$ \rho =\frac{Q}{\frac{4}{3}\pi R^3} \text{ für }r<R $
$ \rho=0 \text{ für }r>R$

\underline{$r<R$}:

$ \phi=\oint_{A}\vec{E}\cdot\mathrm{d}\vec{A} = E(r)\cdot4\pi r^2=\frac{Q_{in}}{\epsilon_0}$

$ Q_{in} = Q\cdot\frac{r^3}{R^3} $

\boxed{$ \implies E(r)=\frac{1}{4\pi\epsilon_0}\cdot\frac{Q}{R^3}\cdot r $}

\hfill\\

\includegraphics{skizzen/14/14_6B7}

\hfill\\

$\implies$ Von Außen ist nicht feststellbar, ob die geladene Kugel massiv oder hohl ist (Leiter oder homogen geladener Isolator)!

\hfill\\

(v) Unendlich lnager homogen geladener Draht

\hfill\\

\includegraphics{skizzen/14/14_6B8}

\hfill\\

Zylinderkoordinaten:

\hfill\\

$ \lambda=\frac{dq}{dL}=(\frac{Q}{R})\leftarrow $ als endliche lange l

$ \vec{E}(\vec{r})=\vec{E}(r)=E(r)\cdot\hat{e}_r $

$ \phi=\oint_{A}\vec{E}\cdot d\vec{A}=\oint_{A}E(r)dA=E(r)\oint_{A}da= E(r)\cdot 2\pi r l $

$ \phi=\frac{Q_{ges}}{\epsilon_0}\implies E(r)= \frac{Q_{ges}}{2\pi\epsilon_0\cdot rl} $

$ E(r)= \underline{ \frac{\lambda}{2\pi\epsilon_0}\cdot\frac{1}{r}}$

\hfill\\

(vi) Unendlich langer, homogen geladener leitender Zylinder

\hfill\\

\includegraphics{skizzen/14/14_6B9}

\hfill\\

(vii) Feld einer homogen geladener unendliche leitenden Ebene

\hfill\\

\includegraphics{skizzen/14/14_6BA}

\hfill\\

$ \sigma=\frac{Q}{A} $

$ \displaystyle\phi = \oint_{A}\vec{E}\cdot d\vec{A}=\int_{A, stirn}^{} \vec{E}\cdot d\vec{A}+\int_{A, mantel}^{} \vec{E}\cdot d\vec{A}   $

$ \vec{E} $-Ebene

$ \implies $ Beitrag Über Mantelfläche verschwindet $ (d\vec{A}\perp\vec{E}) $

$ \phi=\oint_{A} \vec{E}d\vec{A}=2\cdot E\cdot A_{stirn}\overset{G.S.}{=}\frac{Q}{\epsilon_0}$ 

$ \underline{\implies E=\frac{\sigma}{2\epsilon_0}} $
































\end{document}
