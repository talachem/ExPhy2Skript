\documentclass[11pt]{article}

\usepackage{amssymb}
\usepackage{amsthm}
\usepackage{amsmath}
\usepackage[utf8]{inputenc}
\usepackage{mathabx}
\usepackage{framed}
\usepackage{booktabs}
\usepackage{mathtools}
\usepackage{hyperref}

\hypersetup
	{ 
		colorlinks=true,       % false: boxed links; true: colored links
%		hidelinks,
		linkcolor=blue,          % color of internal links (change box color with linkbordercolor)
		citecolor=green,        % color of links to bibliography
		filecolor=magenta,      % color of file links
		urlcolor=cyan,           % color of external links
		linkbordercolor	= {1 0 0},
		citebordercolor	= {0 1 0},	
		urlbordercolor	= {0 1 1}
	}


\usepackage{fontspec}
%\setmainfont{Clear Sans}
\newfontfamily{\clearsans}{Clear Sans}

\newcommand{\definition}{\\ \textbf{Definition:} \hspace{1cm} }

\newcommand*{\QEDA}{\hfill\ensuremath{\blacksquare}}%
\newcommand*{\QEDB}{\hfill\ensuremath{\square}}%


\begin{document}
	\title{Kapitel 14 - Statische elektrische Felder}
		\author
		{Johannes Bilk
			\\
			{\small 	\texttt{me@talachem.de}}
		}
		\date{\today}
	\maketitle
	\tableofcontents
	\setcounter{section}{13} %Hier fängt die Nummerierung an.
	
	\newpage
	
\section{Statische Elektrische Felder}	
	\subsection{ Elektrische Ladungen }	
	
	$\rightarrow$ Ab dem 17. Jahrhundert: Ursache für "elektrische Phänomene"; "neuartiger Stoff", elektrische Ladung
	
		\subsubsection{ Reibungselektrizizät }
		
			\begin{itemize}
			\item Zwei Arten von "elektrischen Zuständen" sind erzeugbar:
				\begin{itemize}
				\item Gleichartige Zustände $\implies$ Abstoßung
				\item Ungleichartige Zustände $\implies$ Anziehung
			\end{itemize}
			\item Carles Du Fay (1730): positiv/negativ elektrische Ladung
			\item Benjamin Franklin (1750): Über-/Unterschuss an "elektrischen Fluiden"
			\item Lichtenberg (1778): Zuordnung der Polariät
			\end{itemize}
			
			\fbox{\begin{minipage}{19em}
			Hargummi stab: reiben mit Pelz, Wolle: -\\
			Glas, Plexiglas: reiben mit Seide: +
			\end{minipage}}
			\linebreak\\
			Reibezeug: entgegengesetzte Polarität
			$\implies$ Ladungstrennung, nicht etwa Ladungserzeugung.
			\linebreak\\
			Grundsätzliches Messprinzip: Elektroskop:
			
			\linebreak\\
			$\rightarrow$ Elektrometer $\rightarrow$ quantitative Messung
			\begin{itemize}
				\item "Löffeln"; d.h. portionsweise Übertragung von Ladungen ist möglich
				\item Elektropendel: $\implies$ periodisches Umladen eines "Kugelpendel"
			\end{itemize}
			
			\subsubsection{Ladung ist eine skalare Größe } Einheit 1C = 1 Coulomb, SI
				\begin{itemize}
					\item Zu jedem geladenen Elementarteilchen gibt es ein Elementarteilchen mit entgegengesetzter Ladung ($\rightarrow$ Ladungssymmetrie)
					\item Die Gesamtladung eines abgeschlossenen Systems bleibt erhalten ($\rightarrow$ Ladungserhaltung)
					\item Beispiel: Produktion eines e^+e^--Paares; E_\gamma \geq 1,02 MeV
				\end{itemize}
				\linebreak\\
				
				\newpage
				
				Nachweis: Blasenkammer im Magnetfeld:
				\linebreak\\
				Umkehrung: "Zerstrahlung" von Positronen; $E=m\cdot c^2$
				\begin{itemize}
					\item Ladungträger haben stets eine Masse
					\item Ladung kann nicht (im Gegensatz zur Masse) in Energie umgewandelt werden, bleibt auch bei Zerfallsprozessen erhalten.
					\item Quantisierung der Ladung: Alle in der Natur vorkommenden Ladungen sind ganzzahlige Vielfache der Elementarladung: $e_0:=1,602\cdot10^{-19}C; 1C=1AS$
				\end{itemize}
				\subparagraph{Beispiele von Ladungen}
				\begin{itemize}
					\item Neutral: $\gamma, \nu, n$
					\item einfach geladen: $e^-,e^+,p, \bar{p}$
					\item zweifach geladen:: $He_2(2^+,Z:2)$
				\end{itemize}	
				
\newpage

\subsubsection{ Quarks }	
\paragraph{Seit 60er Jahre}
Nukleonen bestehen aus Quarks, diese haben "drittelzahlige Ladungen"
\\
Up-Quarks:$u:+\frac{2}{3	}e_0$
\\
Down-Quarks:$d:-\frac{1}{3}e_0$
\\
Proton:$2u+d: 1\cdot e_0$
\\
Neutron:$u+2d: 0\cdot e_0$
\\

Quarks treten immer in 2er- oder 3er- Kombinationen auf.
\\

\subsubsection{Entdeckung und Bestimmung der Elementarladung}
\\
Robert Andrews Millikan(1868-1953): Öltröpfchenversuch ($\rightarrow$ Anfängerpraktikum)

\subsection{Kräfte zwischen Ladungen und das Coulomb-Gesetz}
\\
Charles-Augustin de Coulomb (1736-1806)
\\
1785: Messung der Kraft zwischen zwei Ladungen als Funktion des Abstands mit Hilfe einer Torsionswaage
\linebreak\\

\boxed{\vec{F_{12}} = f\cdot\frac{Q_1\cdot Q_2}{r^2_{12}}\cdot\frac{\vec{r_{12}}}{|\vec{r_{12}}|} = f\cdot\frac{Q_1\cdot Q_2}{r^2_{12}}\cdot \hat{r}_{12}}

\linebreak\\
F ist definiert durch die Definition der Ladungseinheit:
\\
Internationales Messsystem (SI): $f=\frac{1}{4\pi\epsilon_0}$
\\
$\epsilon_0=8,854\cdot10^{-12}\frac{(As)^2}{Nm^2}$
\\
ist Dielektrizitätskonste des Vakuums oder elektrische Feldkonstante
\\
$Q_1\cdot Q_2 > 0:$ Abstoßung
\\
$Q_1\cdot Q_2 < 0:$ Anziehung

\end{document}
