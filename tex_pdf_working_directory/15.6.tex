\subsection{Stromquellen, elektromotorische Kraft, Urspannung, Klemmspannung}

\begin{center}
	\includegraphics[width=0.7\linewidth]{skizzen/15/15_6/15_6B0}
\end{center}

offenes galvanisches Element

\begin{center}
\includegraphics[width=0.4\linewidth]{skizzen/15/15_6/15_6B1}
\end{center}


\underline{innen}

\begin{center}
\includegraphics[width=0.5\linewidth]{skizzen/15/15_6/15_6B2}
\end{center}


\underline{außen}

\begin{center}
	\includegraphics[width=0.5\linewidth]{skizzen/15/15_6/15_6B3}
\end{center}

$$ U_r = \phi_A-\phi_B = \underbrace{(\phi_A - \phi_{el}) + (\phi_{el}-\phi_B)}_{U_0\text{: Urspannung (Elektromotorische Kraft)}}  = U_0 $$

offenes galvanisches Element $ U_R = U_0 $

\hfill\break

Belastet

\begin{center}
	\includegraphics[width=0.4\linewidth]{skizzen/15/15_6/15_6B4}\\
	$ R_i $: Innenwiderstand
\end{center}



\underline{innen}

\begin{center}
	\includegraphics[width=0.5\linewidth]{skizzen/15/15_6/15_6B5}
\end{center}

\underline{außen}

\begin{center}
	\includegraphics[width=0.5\linewidth]{skizzen/15/15_6/15_6B6}
\end{center}

$$ U_r = \phi_A-\phi_B = (\phi_A - \phi_{el}) - I \cdot R_i + (\phi_{el}-\phi_B) $$
$$ \boxed{U_R = U_0 - I \cdot R_i} $$

\begin{center}
	\includegraphics[width=0.5\linewidth]{skizzen/15/15_6/15_6B7}\\
	Kurzschlussstromstärke $ I_R = \frac{U_0}{R_I} $
\end{center}

Im Schaltbild:

\begin{center}
\includegraphics[width=0.6\linewidth]{skizzen/15/15_6/15_6B8}
\end{center}
