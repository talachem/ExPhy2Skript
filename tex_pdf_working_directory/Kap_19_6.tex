\subsection{Wellenausbreitung mit bewegtem Sender/Empfänger} \enter
\subsubsection{Experiment: WWW mit bewegter Quelle}
\bild
Unterschied in $ \lambda $, je nachdem ob sich die Quelle zum Empfänger oder von ihm weg bewegt (Doppler-Effekt).\\
$ \Rightarrow $ Frequenzänderung $ \Delta\nu = \nu-\nu_0 $ Doppler-Verschiebung \\
\begin{align*}
\nu_0 &= \frac{1}{T_0} \text{ : Frequenz der ruhenden Quelle ($ \rightarrow $ Tropfen) }\\
v_Q &: \text{ Geschwindigkeit der Quelle }\\
v_{Ph} &: \text{ Phasengeschwindigkeit der Welle }\\
v_{Ph} &= \frac{\lambda_0}{T_0} = \nu_0 \cdot \lambda_0
\end{align*}
\begin{align*}
\nu_2'&=\frac{v_{Ph}}{\lambda_2} = \frac{v{Ph}}{(v_{Ph}-v_Q)\cdot T_0} = \frac{1}{1-\frac{v_Q}{v_{Ph}}} \cdot \nu_0 \overset{v_Q<<v_{Ph}}{=} (1+\frac{v_Q}{v_{Ph}}) \cdot \nu_0 \hspace{1cm}, \frac{1}{1-x} \overset{x<<1}{\approx} 1+x \\
\nu_1'&=\frac{v_{Ph}}{\lambda_1} = \frac{v{Ph}}{(v_{Ph}+v_Q)\cdot T_0} = \frac{1}{1+\frac{v_Q}{v_{Ph}}} \cdot \nu_0 \overset{v_Q<<v_{Ph}}{=} (1-\frac{v_Q}{v_{Ph}}) \cdot \nu_0
\end{align*}
\begin{align*}
\text{Alternativ: } &v_Q > 0 : \text{ Quelle } \rightarrow \leftarrow \text{ Empfänger } \Rightarrow \nu_2 = \nu_0\\
&v_Q<0: \text{ Quelle } \leftarrow \rightarrow \text{ Empfänger } \Rightarrow \nu_1<\nu_0
\end{align*}
\subsubsection{Bewegter Empfänger}
Bewegt sich E mit $ v_E $ , so wird die effektive Ausbreitungsgeschwindigkeit der Welle geändert:\\

(i) Q $ \rightarrow $ $ \leftarrow $ E :\\
$$ \nu'  = \frac{v_{Ph}+v_E}{\lambda_0} = \frac{v_{Ph}(1+\frac{v_E}{v_{Ph}})}{\lambda_0}=v_0(1+\frac{v_E}{v_{Ph}})$$\\
ist die Geschwindigkeit der Bewegung klein, so ist das Ergebnis für bewegte Quelle und bewegter Empfänger gleich.\\
Beides bewegt sich: \bild
\begin{align*}
\nu'=\frac{v_{Ph}-v_E}{v_{Ph}-v_Q} \cdot \nu \\
v_Q,v_E << v_{Ph}\\
\nu'=\frac{1-\frac{v_E}{v_{Ph}}}{1-\frac{v_Q}{v_{Ph}}} \cdot \nu &= \left[ (1-\frac{v_E}{v_{Ph}}) (1+\frac{v_Q}{v_{Ph}})  \right] \cdot \nu \\
&= \left[ (1  \right] \cdot \nu
\end{align*}
\bild
Beispiel: Polizeisirene: $ v_{Ph} = \SI{300}{\meter\per\second} $ ; $ v_................................ $ 
\rule{5cm}{.2pt}\\
Unterschied zu elektromagnetischer Wellen: Lichtgeschwindigkeit ist konstant.\\
Grenzfall des Doppler-Effektes: $ .... $\\
....
\bild
......
\bild
Im Zeitraum $ t $:\\
Quelle bewegt sich um A nach B \hspace{1cm} .........\\
.....\\
2. Kreiswellen, emittiert zur Zeit $ t_2 $ : $ 0<t_2<t $

$$ ....... $$\\
...............\\
..........\\
$ \Rightarrow $ Schall:
\bild
Zusätzlich "Wolkenscheibeneffekt"
Adiabatische Expansion nach starker Kompression $ \Rightarrow $ Reduzierung der Temperatur............
\bild
\rule{5cm}{.2pt}

\section{20 Wellenausbreitung in 2 und 3 Dimensionen} \hfill\\
Definition: Wellenfront: Punkte gleicher Phase, die zur gleichen Zeit durch Welle angeregt werden. \\
.............\\
$ \Rightarrow $ Jeder Punkt einer Wellenfront ist Erreger einer Kugelförmigen Elementarwelle. Die Einhüllende aller dieser Elementarwellen bildet die Wellenfront zu einem späteren Zeitpunkt als Superposition alle Elementarwellen unter Berücksichtigung ihrer Phase.
\bild
\bild
Resultat: "Beugung" von Wellen \\
Bewegung ist typischer Wellenphänomen, das die Ausbreitung in Bereiche beschreibt, die bei gradlinieger Ausbreitung nicht erreicht werden Können ("geometrischer Schattenbereich")
\bild
Wenn $ \lambda << d $ (d: typ. geometrische Dimension der Hindernisse......\\
...........\\
.......\\
Ist $ \lambda \approx d$ , so müssen dominant typische Wellenphänomene berücksichtigt werden.\\
Reflexion von Wellen
\bild
\kommentar{Wellenfront erreicht Hindernis; Aussenden neuer Elementarwellen; Durch Einhüllende ist die neue Wellenfront}
Geometrie: Einfallswinkel $ \alpha $ $ = $ Ausfallwinkel $ \alpha' $\\
(Aussage nur über Richtung, nicht über Amplitude (Intensität))\\
Brechung: Wellenfront von Medium 1 und $ v_{Ph} = c_1 $ in Medium 2 mit $ v_{Ph} = c_2 $
\bild
In der Zeit $ \tau $ : \begin{align*}
\lambda &= c_1 \cdot \tau \hspace{5mm} \text{Medium 1}\\
\lambda' &= c_2 \cdot \tau \hspace{5mm} \text{Medium 2}\\
\hfill\\
\frac{\lambda}{d} = \sin\alpha& \text{ ; } \frac{\lambda'}{d} = \sin\beta\\
\frac{\lambda}{\sin\alpha} =& \frac{\lambda'}{\sin\beta}\\
\Rightarrow \frac{c_1 \cdot \tau}{\sin\alpha} &= \frac{c_2 \cdot \tau}{\sin\beta} \Leftrightarrow \boxed{..........}
\end{align*}
...........\\
...........\\
\begin{math}
\underline{\frac{\sin\alpha_1}{\sin\alpha_2} = \frac{\lambda_1}{\lambda_2} = \frac{c_1}{c_2}} \hspace{2cm} \underbrace{c_1}_{\text{dünneres Medium}} > \underbrace{c_2}_{\text{dickeres Medium}}
\end{math}
\\
\begin{center}
	\rule{5cm}{.2pt}
\end{center}
\bdt{Interferenz (Experiment: Interferenz Kohärenter Wasserwellen) \\
	Kohärenz: Zwei Wellensysteme sind dann kohärent, wenn sich ihre Phasenbeziehung als Funktion der Zeit nicht ändert, d.h. die Phasendifferenz $ \Delta\varphi $ ist an jedem Raumpunkt zeitlich Konstant und ergibt sich direkt aus dem Laufzeitunterschied.}