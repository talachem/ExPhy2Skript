<<<<<<< Updated upstream
\documentclass[11pt]{article}

\usepackage[utf8]{inputenc}
\usepackage[T1]{fontenc}

\usepackage{amssymb,amsmath,amsthm,mathtools,mathabx,hyperref,txfonts,siunitx,booktabs,fontspec}


\hypersetup
	{ 
		colorlinks=true,       % false: boxed links; true: colored links
		linkcolor=blue,          % color of internal links (change box color with linkbordercolor)
		citecolor=green,        % color of links to bibliography
		filecolor=magenta,      % color of file links
		urlcolor=cyan,           % color of external links
		linkbordercolor	= {1 0 0},
		citebordercolor	= {0 1 0},	
		urlbordercolor	= {0 1 1}
	}

\newcommand{\definition}{\\ \textbf{Definition:} \hspace{1cm} }

\newcommand*{\QEDA}{\hfill\ensuremath{\blacksquare}}%
\newcommand*{\QEDB}{\hfill\ensuremath{\square}}%


\begin{document}
	\title{Experimental Physik II Kapitel 16}
		\author
			{
				author\\
				{\small 	\texttt{email}}
			}
		\date{\today}
	\maketitle
	\tableofcontents
	\setcounter{section}{15} %Hier fängt die Nummerierung an.
	\newpage
	
\section{Statische magnetische Felder}

\end{document}
=======
\documentclass[11pt]{article}

\usepackage[utf8]{inputenc}
\usepackage[T1]{fontenc}

\usepackage{amssymb}
\usepackage{amsthm}
\usepackage{amsmath}
\usepackage{mathtools}
\usepackage{mathabx}
%\usepackage{framed}
\usepackage{booktabs}
\usepackage{hyperref}
\usepackage{txfonts}
\usepackage{siunitx}
\hypersetup
	{ 
		colorlinks=true,       % false: boxed links; true: colored links
%		hidelinks,
		linkcolor=blue,          % color of internal links (change box color with linkbordercolor)
		citecolor=green,        % color of links to bibliography
		filecolor=magenta,      % color of file links
		urlcolor=cyan,           % color of external links
		linkbordercolor	= {1 0 0},
		citebordercolor	= {0 1 0},	
		urlbordercolor	= {0 1 1}
	}


%\usepackage{fontspec}
%\setmainfont{Clear Sans}
%\newfontfamily{\clearsans}{Clear Sans}

\newcommand{\definition}{\\ \textbf{Definition:} \hspace{1cm} }

\newcommand*{\QEDA}{\hfill\ensuremath{\blacksquare}}%
\newcommand*{\QEDB}{\hfill\ensuremath{\square}}%


\begin{document}
	\title{Titel}
		\author
		{
			author\\
			{\small 	\texttt{email}}
		}
		\date{\today}
	\maketitle
	\tableofcontents
	\setcounter{section}{15} %Hier fängt die Nummerierung an.
	
	\newpage
	
\section{Statische magnetische Felder }
\paragraph{Experimente:}
\begin{itemize}
	\item gleichnamige Pole stoßen sich ab
	\item ungleichnamige Pole ziehen sich an\\
	\item Kraftwirkung $\propto\frac{1}{r^2}$ (1750; Coulomb)
	\item ähnliche Abstandsabhängigkeit für elektrische und für magnetische Kräfte
	\item zunächst kein Zusammenhang zwischen beiden Kräften erkennbar
	\item Experiment: Magnetische Pole treten nur paarweise auf. \\ ($\implies$ keine "magnetische Ladung")
\end{itemize}

\paragraph{Feldlinien sichtbarmachen durch Eisenfeilspitzen:}\leavevmode \\

\boxed{Magnetische Feldlinien sind stets geschlossen; es gibt keine isolierbaren Quellen oder Senken des magnetischen Felds.}

\paragraph{Erinnerung: Satz von Gauß:}\leavevmode \\
\includegraphics{skizzen/16/16_0B01}

$\vec{E}:$ elektrische Feldstärke:\\
Gesamtfluss: $\phi_{el}=\oint_A \vec{E}\cdot d\vec{A}=\frac{Q}{\epsilon_0}$\\

\subparagraph{Magnetische Felder:}\leavevmode \\

Gesamtfluss: $\boxed{ \phi_{mag}=\oint_A \underbrace{\vec{B}\cdot d\vec{A}}_{\text{magnetischer Fluss}}=0 \\ \vec{B}: \text{magnetische Flussdichte}}$
	
	\subsection{ Kräfte auf bewegte Ladungen }	
		\subsubsection{ Lorentzkraft $\vec{F}_L$ }
		$\vec{F}_L = q\cdot\vec{v}\times\vec{B}\\ (\vec{F}_L \perp \vec{v}; \vec{F}\perp \vec{B} ) $\\
		
		\includegraphics{skizzen/16/16_1B01}\\
		
		UVW-Regel: Ursache $\rightarrow$ Vermittler $\rightarrow$ Wirkung\\
		Vorsicht!: Elektrische Ladung ist negativ!\\
		
		$[\vert\vec{B}\vert]=\frac{N}{As\cdot\frac{m}{s}}=\frac{Vs}{m^2}=1T (Tesla)$
		
		Kreisbahn: $\vec{F}_L\perp \vec{v}$
		
		$\implies \vec{F}_L$ beeinfluss die Richtung von $\vec{v}$, aber nicht den Betrag!\\
		$\implies \vec{F}_L$ leistet keine Arbeit
		
			\paragraph{Konventionen:}\leavevmode \\
			
			\includegraphics{skizzen/16/16_1B02}\\
			
			$\otimes \vec{B}$ zeigt in die Papierebene hinein\\
			
			$\odot \vec{B}$ zeigt aus der Papierebene heraus\\
			
			
	\subsubsection{Bewegungsgleichung:}\leavevmode \\
	
	$m\ddot{\vec{r}} = \dot{\vec{r}}=\vec{F}_L=q\cdot\vec{v}\times\vec{B} $\\
	
	$\frac{d\vec{v}}{dt}=\dot{\vec{v}}=\frac{\dot{\vec{p}}}{m}=\frac{q}{m}\cdot\vec{v}\times\vec{B}$\\
	
	$d\vec{v}\perp \vec{v}; d\vec{v}\perp\vec{B}$\\
	
	$\implies$	Kreisbahn: $\vec{F}_L$ ist Zentripetalkraft\\
	
	$\implies q\cdot v\cdot B=m\cdot\frac{v^2}{r}; v=\omega\cdot r$\\
	
	$\boxed{\omega=\frac{q}{m}\cdot B}$\\
	
	$\boxed{\nu=\frac{1}{2\pi}\cdot\frac{q}{m}\cdot B}$\\
	
	$\omega$ Zyklotronfrequenz (1930, Lawrence)\\
	
	$\implies$ unabhängig von Impuls und Energie; nur von $\frac{q}{m}$ und $\vec{B}$ bestimmt!\\
	
	\paragraph{Radius:}\leavevmode \\
	
	$r=\frac{m\cdot v}{q\cdot B}=\frac{p}{q\cdot B}= \frac{\sqrt{2mqV}}{q\cdot B}$\\
	
	$E_{kin}=\frac{p^2}{2m}=\frac{1}{2}m\cdot v^2=q\cdot V$\\
	
	\subparagraph{Experiment:}\leavevmode \\
	
	$r_1: V_1=200V\implies 2SKT$\\
	
	$r_: V_1=300V\implies 2,5SKT$\\
	
	$\frac{r_1}{r_2}\overset{!}{=} \sqrt{\frac{V_1}{V_2}}$\\
	
	$\frac{4}{5}\overset{!}{=}\sqrt{\frac{2}{3}}$\\
	
	$\frac{16}{25}\overset{!}{=}\frac{2}{3}$ \checkmark im Rahmen der Messungenaugikeit!
	
				
\newpage	
	
% Unnummerierte Struktur:

\phantomsection
\addcontentsline{toc}{section}{Unnummerierte Section}
\section*{SECTION}
	\phantomsection
	\addcontentsline{toc}{subsection}{Unnummerierte subsection}
	\subsection*{ subsection }
		\phantomsection
		\addcontentsline{toc}{subsubsection}{Unnummerierte subsubsection}
		\subsubsection*{ subsubsection }
			\paragraph{paragraph z.B. Definition/Exp/Beispiele/Anwendung }
				\subparagraph{Beispiel 1/3, Exp 1/2, Fallunterscheidungen}	

	

\end{document}
>>>>>>> Stashed changes
