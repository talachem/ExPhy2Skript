\documentclass[11pt]{article}

\usepackage{amssymb}
\usepackage{amsthm}
\usepackage{amsmath}
\usepackage{mathtools}
\usepackage[utf8]{inputenc}
\usepackage{mathabx}
%\usepackage{framed}
\usepackage{booktabs}
\usepackage{hyperref}
\usepackage{txfonts}
\hypersetup
	{ 
		colorlinks=true,       % false: boxed links; true: colored links
%		hidelinks,
		linkcolor=blue,          % color of internal links (change box color with linkbordercolor)
		citecolor=green,        % color of links to bibliography
		filecolor=magenta,      % color of file links
		urlcolor=cyan,           % color of external links
		linkbordercolor	= {1 0 0},
		citebordercolor	= {0 1 0},	
		urlbordercolor	= {0 1 1}
	}


%\usepackage{fontspec}
%\setmainfont{Clear Sans}
%\newfontfamily{\clearsans}{Clear Sans}

\newcommand{\definition}{\\ \textbf{Definition:} \hspace{1cm} }

\newcommand*{\QEDA}{\hfill\ensuremath{\blacksquare}}%
\newcommand*{\QEDB}{\hfill\ensuremath{\square}}%


\begin{document}
	\title{Titel}
		\author
		{
			author\\
			{\small 	\texttt{email}}
		}
		\date{\today}
	\maketitle
	\tableofcontents
	\setcounter{section}{14} %Hier fängt die Nummerierung an.
	
	\newpage
	
\section{SECTION }	
	\subsection{ subsection }	
		\subsubsection{ subsubsection }
			\paragraph{paragraph z.B. Definition/Exp/Beispiele/Anwendung }
				\subparagraph{Beispiel 1/3, Exp 1/2, Fallunterscheidungen}	
				
\newpage	
	
% Unnummerierte Struktur:

\phantomsection
\addcontentsline{toc}{section}{Unnummerierte Section}
\section*{SECTION}
	\phantomsection
	\addcontentsline{toc}{subsection}{Unnummerierte subsection}
	\subsection*{ subsection }
		\phantomsection
		\addcontentsline{toc}{subsubsection}{Unnummerierte subsubsection}
		\subsubsection*{ subsubsection }
			\paragraph{paragraph z.B. Definition/Exp/Beispiele/Anwendung }
				\subparagraph{Beispiel 1/3, Exp 1/2, Fallunterscheidungen}	

	

\end{document}
