\documentclass[11pt]{article}

\usepackage[utf8]{inputenc}
\usepackage[T1]{fontenc}

\usepackage[dvipsnames]{xcolor}
\usepackage{blindtext,tcolorbox,graphicx,color}
\usepackage[makeroom]{cancel}
\usepackage{amssymb,amsmath,amsthm,mathtools,mathabx,hyperref,txfonts,siunitx,booktabs,fontspec}

%Beispiel für farbige boxen:

%\begin{tcolorbox}[width=\textwidth,colback={green},title={With rounded corners},colbacktitle=yellow,coltitle=blue]    
%	\blindtext[1]
%\end{tcolorbox}    
%
%\begin{tcolorbox}[width=\textwidth,colback={red},title={With true corners},outer arc=0mm,colupper=white]    
%	\blindtext[1]     
%\end{tcolorbox} 

\hypersetup
{ 
	colorlinks=true,       % false: boxed links; true: colored links
	linkcolor=blue,          % color of internal links (change box color with linkbordercolor)
	citecolor=green,        % color of links to bibliography
	filecolor=magenta,      % color of file links
	urlcolor=cyan,           % color of external links
	linkbordercolor	= {1 0 0},
	citebordercolor	= {0 1 0},	
	urlbordercolor	= {0 1 1}
}

\newcommand{\definition}{\\ \textbf{Definition:} \hspace{1cm} }

\newcommand*{\QEDA}{\hfill\ensuremath{\blacksquare}}%
\newcommand*{\QEDB}{\hfill\ensuremath{\square}}%
\newcommand*{\ds}[1]{{\displaystyle #1}}
\newcommand*{\bild}{\begin{center} \textbf{BILD fehlt hier noch} \end{center}}
\newcommand*{\erg}[2]{\begin{tcolorbox}[width=\textwidth,colback={Orchid},title={#1},colbacktitle=SkyBlue,coltitle=Black] #2  \end{tcolorbox}}
\newcommand{\enter}{\hfill\\}
\newcommand{\abs}[1]{\left| #1 \right| }
\newcommand{\HL}{\begin{center} \rule{5cm}{.2pt} \end{center}}
\newcommand{\bdt}[1]{\begin{tcolorbox}[width=\textwidth,colback={White}] #1  \end{tcolorbox}}
\newcommand*{\kommentar}[1]{\begin{tcolorbox}[width=\textwidth,colback={Lavender}] #1  \end{tcolorbox} }


\begin{document}
	\title{Titel}
		\author
		{
			author\\
			{\small 	\texttt{email}}
		}
		\date{\today}
	\maketitle
	\tableofcontents
	\setcounter{section}{14} %Hier fängt die Nummerierung an.
	
	\newpage
	
\section{SECTION }	
	\subsection{ subsection }	
		\subsubsection{ subsubsection }
			\paragraph{paragraph z.B. Definition/Exp/Beispiele/Anwendung }
				\subparagraph{Beispiel 1/3, Exp 1/2, Fallunterscheidungen}	
				
\newpage	
	
% Unnummerierte Struktur:

\phantomsection
\addcontentsline{toc}{section}{Unnummerierte Section}
\section*{SECTION}
	\phantomsection
	\addcontentsline{toc}{subsection}{Unnummerierte subsection}
	\subsection*{ subsection }
		\phantomsection
		\addcontentsline{toc}{subsubsection}{Unnummerierte subsubsection}
		\subsubsection*{ subsubsection }
			\paragraph{paragraph z.B. Definition/Exp/Beispiele/Anwendung }
				\subparagraph{Beispiel 1/3, Exp 1/2, Fallunterscheidungen}	

%Mit \include{DATEINAME} werden tex dateien eingebunden.	

\end{document}
